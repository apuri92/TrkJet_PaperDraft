% !TEX root = trkjet.tex

The \PbPb\ and \pp\ data used in this analysis were recorded in 2015.
The data samples consist of 25~pb$^{-1}$ of \sqrts~=~5.02~\TeV\ \pp\ and 0.49~nb$^{-1}$ of \sqrtsnn~=~5.02~\TeV\ \pbpb\ data.
In both samples, events are required to have a reconstructed vertex within 150~mm of the nominal IP along the beam axis.
The pileup is negligible in the \pbpb\ while the \pp\ data is collected in low pileup mode.
The average number of interactions per bunch crossing in \pp\ collisions ranges from 0.6 to 1.3.
Only events taken during stable beam conditions and satisfying detector and data-quality requirements that include the detector subsystems being in nominal operating conditions are considered.

The \pp\ Monte Carlo (MC) used a set of $1.8\times10^7$ 5.02 TeV hard-scattering dijet \pp\ events generated with \powheg{}+\pythiaeight\ \cite{Nason:2004rx,Sjostrand:2014zea} using the A14 tune of parameters \cite{ATLAS2014021} and the NNPDF23LO PDF set \cite{Ball:2012cx}.

In \pbpb\ collisions, the event centrality reflects the overlap area of the two colliding nuclei and is characterized by \ETfcal, the total transverse energy deposited in the FCal~\cite{Aaboud:2017tql}.
The six centrality intervals used in this analysis are defined according to successive percentiles of the \ETfcal\ distribution obtained in minimum bias (MB) collisions, ordered from the most central (highest \ETfcal) to the most peripheral (lowest \ETfcal) collisions: 0--10\%, 10--20\%, 20--30\%, 30--40\%, 40--60\%, 60--80\%.

The \pbpb\ MC was generated by overlaying an additional sample of MB \pbpb\ data events on a separate set of $1.8\times10^7$ 5.02 TeV hard-scattering dijet \pp\ events generated with the same tune and PDFs as the \pp\ MC.
This ``MC overlay'' sample is reweighted on an event-by-event basis such that it has the same centrality distribution as the jet triggered sample.
Another sample of MB \pbpb\ events was generated using HIJING (version 1.38b) \cite{Wang:1991hta} and was only used to evaluate the track reconstruction performance.

The detector response was simulated in all MC samples using \textsc{Geant4} \cite{Agostinelli:2002hh,Aad:2010ah}. These MC samples are used to evaluate the performance of the detector and analysis procedure and correct the measured distributions for detector effects.



