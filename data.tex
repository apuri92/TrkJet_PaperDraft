% !TEX root = trkjet.tex

The \PbPb\ and \pp\ data used in this analysis were recorded in 2015.
 The data samples consist of 25~pb$^{-1}$ of \sqrts~=~5.02~\TeV\ \pp\ and 0.49~nb$^{-1}$ of \sqrtsnn~=~5.02~\TeV\
\pbpb\ data. In both samples, events are required to have a reconstructed vertex
within 150~mm of the nominal IP along the beam axis.
Only events taken during stable beam conditions and satisfying detector and data-quality requirements that include the calorimeters and inner tracking detectors being in nominal operating conditions are considered. 

%The MB
%\PbPb\ sample was recorded with different pre-scales\footnote{The pre-scale indicates which fraction of events that passed the trigger selection was selected for recording by the data  acquisition.}  depending on the instantaneous luminosity in the
%LHC fill. The MB trigger recorded an effective luminosity of 22 $\mu$b$^{-1}$.


In \PbPb\ collisions, the event centrality reflects the overlap area of the two colliding nuclei and is characterized by \ETfcal, the total transverse energy deposited in the 
FCal~\cite{Aaboud:2017tql}. The six centrality intervals used in this analysis are defined according to successive percentiles of the \ETfcal\ distribution obtained in MB collisions, ordered from the most central (highest \ETfcal) to the most peripheral (lowest \ETfcal) collisions: 0--10\%, 10--20\%, 20--30\%, 30--40\%, 40--60\%, 60--80\%. 

In addition to the jet-triggered sample, a separate MB \PbPb\ data sample was recorded with three trigger selections: the minimum-bias trigger and two total transverse-energy triggers with thresholds of 1.5 TeV and 6.5 TeV used to enhance the rate of central \pbpb\ events. This sample was combined with a set of $1.8\times10^7$ 5.02 TeV hard-scattering dijet \pp\ events generated with \powheg{}+\pythiaeight\ \cite{Nason:2004rx,Sjostrand:2014zea} using the A14 tune of parameters \cite{ATLAS2014021} and the NNPDF23LO PDF set \cite{Ball:2012cx} to produce the Monte Carlo Overlay sample. This sample is reweighed on an event-by-event basis such that it has the same centrality distribution as the jet triggered sample.
A separate set of $1.8\times10^7$ 5.02 TeV hard-scattering dijet \pp\ events generated with the same tune and PDFs was used as the \pp\ MC. The detector response was simulated in both MC samples using \textsc{Geant4} \cite{Agostinelli:2002hh,Aad:2010ah} and was used to evaluate the performance of the detector and analysis procedure. Another sample of MB \pbpb\ events was generated using HIJING (version 1.38b) \cite{Aad:2010ah} and was only used to evaluate the track reconstruction performance. 	


%A sample of 1.8~$\times10^{7}$ simulated 5.02~\TeV\ \powheg{}+\pythiaeight~\cite{Nason:2004rx,Sjostrand:2014zea} \pp\ hard-scattering events, generated using the A14 tune~\cite{ATLAS2014021} and the NNPDF23LO PDF set~\cite{Ball:2012cx}, is used to evaluate the performance for measuring \Dptr\ distributions in the \pp\ data. The performance of the detector and analysis procedure in \PbPb\ collisions is evaluated using 1.8~$\times10^{7}$ 5.02~\TeV\ hard-scattering dijet events generated with \powheg{}+\pythiaeight\ overlaid on top of events from the enhanced minimum-bias \PbPb\ data sample. In both samples, the detector response is simulated using \textsc{Geant}4~\cite{Agostinelli:2002hh,Aad:2010ah}. A weight is assigned to each MC event such that the event sample obtained from the simulation has the same \ETfcal\ distribution as in jet triggered data.


