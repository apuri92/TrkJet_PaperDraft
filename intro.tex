% !TEX root = trkjet.tex

Ultra-relativistic nuclear collisions at the Large Hadron Collider (LHC) produce hot, dense matter
called the quark-gluon plasma, QGP (see Refs.~\cite{Roland:2014jsa,Busza:2018rrf} for recent reviews).
Hard-scattering processes occurring in these collisions produce jets which
traverse and interact with the QGP. The comparison between the rates and the characteristics of these jets when produced in heavy-ion or \pp\ collisions provides information on the properties of the QGP and how it interacts with partons from the hard scattering.

Jets with large transverse momenta are observed to be produced in central lead-lead (\pbpb) collisions at the LHC at a rate that is reduced by a factor of two with respect to the expectation from these cross sections measured in \pp\ interactions, re-scaled by the nuclear overlap function of \pbpb\ collisions~\cite{Abelev:2013kqa,Aad:2014bxa,Khachatryan:2016jfl, 2019108}. 
%The rates of jet production are observed to be reduced by  approximately a factor of two in lead-lead~(\PbPb) collisions at LHC energies compared to  expectations from the jet production cross sections measured in \pp\ interactions scaled by the nuclear overlap function of \PbPb\ collisions~\cite{Abelev:2013kqa,Aad:2014bxa,Khachatryan:2016jfl}. 
%This reduction is termed ``jet-quenching'' and is due to the interaction of
%constituents of the parton shower with the QGP.  
Similarly, back-to-back dijet~\cite{Aad:2010bu,Chatrchyan:2011sx,Aaboud:2017eww} 
and photon-jet pairs~\cite{Chatrchyan:2012gt} are observed to have
unbalanced transverse momenta in \pbpb\ collisions compared to \pp\ collisions.
These observations suggest that some of the energy from the hard-scattered parton is
transferred outside of the jet through its interaction with the QGP.  

Also of interest are measurements sensitive to the distributions of particles
within the jet.  Measurements of the jet shape~\cite{Chatrchyan:2013kwa} and  the longitudinal fragmentation functions~\cite{Aad:2014wha,Chatrchyan:2014ava,Aaboud:2017bzv, PhysRevC.98.024908} were performed in 2.76~\TeV\ and 5.02~\TeV\ \pbpb\
collisions.
These measurements indicate that \pbpb\ collisions have an excess of low and high momentum particles along with a depletion of intermediate momentum particles inside the jet compared to \pp\ collisions. Particles carrying a large fraction of the jet momentum are generally closely
aligned with the jet axis, whereas low momentum particles can have a much broader
angular distribution extending outside the jet \cite{Khachatryan:2016tfj,Sirunyan:2018jqr}. 
These observations suggest that the energy lost by jets as they propagate through the QGP (a process called jet-quenching) is being transferred to soft particles within and around the jet~\cite{Qin:2015srf,Blaizot:2014ula}. Measurements of yields of these particles as a function of transverse momentum and
angular distance between the particle and the jet axis have a potential to provide additional constraints on models of jet energy loss processes in \pbpb\ collisions.

This paper presents a measurement of charged particle \pt\ distributions inside and around jets. The measured yields are defined as:
  \begin{equation}
  \Dptr = \frac{1}{N_{\mathrm{jet}}} \frac{1}{\mathrm{A}} \frac{\mathrm{d} n_{\mathrm{ch}} (\pt, r)}{\mathrm{d} \pt},
%D(\pt,\ptjet) = \frac{1}{N_{\mathrm{jet}}} ~ \frac{1}{\epsilon(\pttrk)} ~ \frac{\mathrm{d} N_{\mathrm{ch}}}{\mathrm{d} \pt}~(\ptjet).
\end{equation}
where $N_{\mathrm{jet}}$ is the number of jets in consideration, $A = \pi (r_{\mathrm{max}}^2 - r_{\mathrm{min}}^2) $ is the area of an annulus around the jet with its inner and outer radii $r_{\mathrm{min}}$ and $r_{\mathrm{max}}$. The angular distance from the jet axis is given by $r = \sqrt{\Delta \eta^2 + \Delta \phi^2}$ \footnote{ATLAS uses a right-handed coordinate system with its origin at the nominal interaction point (IP) in the center of the detector, and the $z$-axis along the beam pipe. The $x$-axis points from the IP to the center of the LHC ring, and the $y$-axis points upward. Cylindrical coordinates $(r, \phi)$ are used the transverse plane, $\phi$ being the azimuthal angle around the $z$-axis. The pseudorapidity is defined in terms of the polar angle $\theta$ as $\eta = - \text{ln} \tan (\theta/2)$. Transverse momentum and transverse energy are defined as $\pt = p \sin\theta$ and $\Et = E \sin\theta$, respectively. $\Delta R = \sqrt{(\Delta \eta )^2 + (\Delta \phi)^2}$ gives the angular distance between two objects with relative differences $\Delta \eta$ and $\Delta \phi$ in pseudorapidity and azimuth respectively.}, and $n_{\mathrm{ch}}(\pt, r)$ is the number of charged particles with a given \pt\ within the annulus. The ratios of the charged-particle yields measured in \pbpb\ and \pp\ collisions,
\begin{equation}
   \RDptr = \frac{\Dptr_\mathrm{Pb+Pb}}{\Dptr_{pp}}
   \label{eq:rdptr}
\end{equation}
allow evaluation of modifications of the yields in \pbpb\ collisions compared to those in \pp\ collisions. The differences between the \Dptr\ distributions in \pbpb\ and \pp\ collisions, 
\begin{equation}
   \Delta \Dptr = \Dptr_\mathrm{Pb+Pb} - \Dptr_{pp}
   \label{eq:deltadptr}
\end{equation}
 are also presented and allow for quantifying the absolute differences between the two collision systems even in regions where the \Dptr\ distributions in \pp\ tend toward 0.

The analysis is done using 0.49~nb$^{-1}$ of \pbpb\ collisions and 
25~pb$^{-1}$ of \pp\ collisions at center-of-mass energy of 5.02~\TeV\ collected in 2015 by ATLAS.
Jets are reconstructed with the \antikt\ algorithm~\cite{Cacciari:2008qp} using a radius parameter \RFour\ over a rapidity range of $|\yjet| <$~1.7. The measurement is presented for jets with transverse momenta \ptjet, in the 126 -- 316~\GeV\ range, for charged particles with $\pT>1.0$~\GeV\ and for the following successive intervals in $r$ around the jet, forming the annuli with inner and outer radii $r_{\textrm{min}}$ and $r_{\textrm{max}}$: 0.0, 0.05, 0.1, 0.15, 0.2, 0.25, 0.3, 0.4, 0.5, 0.6, 0.7, 0.8

