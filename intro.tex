% !TEX root = trkjet.tex

Ultra-relativistic nuclear collisions at the Large Hadron Collider (LHC) produce hot, dense matter
called the quark-gluon plasma, QGP (see Refs.~\cite{Roland:2014jsa,Busza:2018rrf} for recent reviews).
Jets from hard-scattering processes in these collisions 
traverse and interact with the QGP, losing energy via a process called jet-quenching. The rates and characteristics of these jets in heavy-ion collisions can be compared to the same quantities in \pp\ collisions, where we do not expect the production of QGP. This comparison can provide information on the properties of the QGP and how it interacts with partons from the hard scattering.

Jets with large transverse momenta in central lead-lead (\pbpb) collisions at the LHC are produced at approximately half the rates measured in \pp\ collisions when the nuclear overlap function of \pbpb\ collisions is taken into account~\cite{Abelev:2013kqa,Aad:2014bxa,Adam:2015ewa,Khachatryan:2016jfl, 2019108}. 
%The rates of jet production are observed to be reduced by  approximately a factor of two in lead-lead~(\PbPb) collisions at LHC energies compared to  expectations from the jet production cross sections measured in \pp\ interactions scaled by the nuclear overlap function of \PbPb\ collisions~\cite{Abelev:2013kqa,Aad:2014bxa,Khachatryan:2016jfl}. 
%This reduction is termed ``jet-quenching'' and is due to the interaction of
%constituents of the parton shower with the QGP.  
Similarly, back-to-back dijet~\cite{Aad:2010bu,Chatrchyan:2011sx,Aaboud:2017eww} 
and photon-jet pairs~\cite{Chatrchyan:2012gt,Aaboud:2018anc} are observed to have
less balanced transverse momenta in \pbpb\ collisions compared to \pp\ collisions.
These observations suggest that some of the energy from the hard-scattered parton may be
transferred outside of the jet through its interaction with the QGP medium.  

Complementary measurements look at how the structure of jets is different between \pbpb\ and \pp\
collisions.
Jet shape measurements in the \pp\ and \pbpb\ collision systems have shown 
a broadening of the jets due to the QGP~\cite{Aad:2011sc, Acharya:2018uvf, Chatrchyan:2012mec, Chatrchyan:2013kwa}.
Additionally, longitudinal fragmentation functions have been measured in 2.76~\TeV\ and 5.02~\TeV\ \pbpb\
collisions;
these indicate that \pbpb\ collisions have an excess of low and high momentum particles along with a depletion of intermediate momentum particles inside the jet compared to \pp\ collisions~\cite{Aad:2014wha,Chatrchyan:2014ava,Aaboud:2017bzv,Aaboud:2018hpb}. Particles carrying a large fraction of the jet momentum are generally closely
aligned with the jet axis, whereas low momentum particles are observed to have a much broader
angular distribution extending outside the 
jet~\cite{Chatrchyan:2011sx,Khachatryan:2015lha,Khachatryan:2016tfj,Sirunyan:2018jqr}. 
These observations suggest that the energy lost via jet-quenching is being transferred to soft particles within and in the vicinity of the 
jet through emission of soft 
gluons~\cite{Vitev:2008rz,Ovanesyan:2011xy,Blaizot:2014ula,Qin:2015srf,Escobedo:2016jbm,Casalderrey-Solana:2016jvj,Tachibana:2017syd}. 
Measurements of yields of these particles as a function of transverse momentum and
angular distance between the particle and the jet axis have a potential to provide further insight
on the structure of jets in the QGP as well as provide constraints on how the medium is affected by the presence of the jet. 

This paper presents a measurement of charged particle \pt\ distributions inside and around jets in different centrality intervals. The measured yields are defined as:

\begin{align*}
\Dptr = \frac{1}{N_{\mathrm{jet}}} \frac{1}{\mathrm{A}} \frac{\mathrm{d} n_{\mathrm{ch}} (\pt, r)}{\mathrm{d} \pt},
%D(\pt,\ptjet) = \frac{1}{N_{\mathrm{jet}}} ~ \frac{1}{\epsilon(\pttrk)} ~ \frac{\mathrm{d} N_{\mathrm{ch}}}{\mathrm{d} \pt}~(\ptjet).
\end{align*}

where $N_{\mathrm{jet}}$ is the number of jets in consideration, $A = \pi (r_{\mathrm{max}}^2 - r_{\mathrm{min}}^2) $ is the area of an annulus around the jet axis with its inner and outer radii $r_{\mathrm{min}}$ and $r_{\mathrm{max}}$ respectively. The angular distance from the jet axis is given by $r = \sqrt{\Delta \eta^2 + \Delta \phi^2}$ \footnote{ATLAS uses a right-handed coordinate system with its origin at the nominal interaction point (IP) in the center of the detector, and the $z$-axis along the beam pipe. The $x$-axis points from the IP to the center of the LHC ring, and the $y$-axis points upward. Cylindrical coordinates $(r, \phi)$ are used the transverse plane, $\phi$ being the azimuthal angle around the $z$-axis. The pseudorapidity is defined in terms of the polar angle $\theta$ as $\eta = - \text{ln} \tan (\theta/2)$. The rapidity is defined as $y = 0.5\text{ln}[(E + p_z)/(E-p_z)]$ where $E$ and $p_z$ are the energy and the component of the momentum along the beam direction.  Transverse momentum and transverse energy are defined as $\pt = p \sin\theta$ and $\Et = E \sin\theta$, respectively. $\Delta R = \sqrt{(\Delta \eta )^2 + (\Delta \phi)^2}$ gives the angular distance between two objects with relative differences $\Delta \eta$ and $\Delta \phi$ in pseudorapidity and azimuth respectively.}, and $n_{\mathrm{ch}}(\pt, r)$ is the number of charged particles with a given \pt\ within the annulus. The ratios of the charged-particle yields measured in \pbpb\ and \pp\ collisions,

\begin{align*}
   \RDptr = \frac{\Dptr_\mathrm{Pb+Pb}}{\Dptr_{\pp}},
\end{align*}

allow evaluation of modifications of the yields from the QGP medium. The differences between the \Dptr\ distributions in \pbpb\ and \pp\ collisions, 

\begin{align*}
   \Delta \Dptr = \Dptr_\mathrm{Pb+Pb} - \Dptr_{pp},
\end{align*}

allow for quantifying the absolute differences in charged-particle yields between the two collision systems. 

The analysis is done using 0.49~nb$^{-1}$ of \pbpb\ collisions and 
25~pb$^{-1}$ of \pp\ collisions at center-of-mass energy of 5.02~\TeV\ collected in 2015 by ATLAS. 

%The \pbpb\ collisions are divided into the following centrality intervals: 0--10\%, 10--20\%, 20--30\%, 30--40\%, 40--60\%, 60--80\%.
%It uses jets reconstructed with the \antikt\ algorithm \cite{Cacciari:2008qp} using a radius parameter of \RFour, restricted to the rapidity interval of $|\yjet| <$~1.7 and having transverse momenta \ptjet\ in the 126--316 GeV range. Charged particles associated with these jets are restricted to $|\eta| < 2.5$ and have a transverse momenta of $\pt > 1$ GeV. The measurement is done in annuli at increasing distances from the jet axis. These annuli have their inner and outer radius $r_{\textrm{min}}$ and $r_{\textrm{max}}$ and take the following values: 0.0, 0.05, 0.1, 0.15, 0.2, 0.25, 0.3, 0.4, 0.5, 0.6, 0.7, 0.8.

