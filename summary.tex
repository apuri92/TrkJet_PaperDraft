% !TEX root = trkjet.tex

This paper presents a measurement of the yields of charged particles, \Dptr, inside and around \RFour\ \antikt\ jets with $|\yjet| <$1.7 up to a distance of $r = 0.8$ from the jet axis.
The yields are measured in intervals of \ptjet\ from 126 to 316~\GeV\ in \PbPb\ and \pp\ collisions at 5.02~\TeV\ as a function of charged-particle \pt\ and the angular distance \rvar\ between the jet axis and charged particle.


These results show a broadening of the \Dptr\ distribution for low \pt\ particles inside the jet in central \pbpb\ collisions compared to those in \pp\ collisions while for higher \pt\ particles angular distributions are narrower in \pbpb\ collisions compared to \pp\ collisions.
These modifications are centrality dependent and decrease for more peripheral collisions.
The \RDptr\ distributions for charged particles with $\pt <$~4~\GeV\ are above unity and grow with increasing angular separation up to $r \sim0.3$, showing weak to no dependence on $r$ in the interval 0.3~$< \rvar <$~0.6 followed with a small decrease in the enhancement for 0.6~$< \rvar <$~0.8.
For charged particles with $\pt >$~4~\GeV, a suppression in \RDptr\ is observed, and the distributions decrease with increasing \rvar\ for 0.05 $ < \rvar < $~0.3, with no \rvar\ dependence for $r>0.3$.
For all charged-particle \pt\ values, the \RDptr\ values are greater than or equal to unity for $\rvar <$~0.05.
Between $0.1 < r < 0.25$, a statistically significant trend of increasing \RDptr\ with increasing \ptjet\ is observed for low-\pt\ particles.
No significant \ptjet\ dependence is seen for particles  with $\pt >$~4~\GeV.

This measurement provides information about the modification of the jet at large distances from the jet axis that can be used to constrain models that distinguish the modifications of jet due to the presence of the plasma from the response of the medium to the jet.




%Centrality dependent modifications to the yields, when compared to those measured in \pp\ collisions, are observed.
%The magnitude of these modifications increases with increasing collision centrality.

%The \RDptr\ distributions for charged particles with $\pt <$~4~\GeV\ 
%are above unity and 
%grow with increasing angular separation up to $r \sim0.3$, showing weak to no dependence on $r$ in the interval 0.3~$< \rvar <$~0.6 followed with a small decrease in the enhancement for 0.6~$< \rvar <$~0.8.
%For charged particles with $\pt >$~4~\GeV, a suppression in \RDptr\ is observed, and the 
%distributions decrease with increasing
%\rvar\ for $\rvar < $~0.3, with no \rvar\ dependence for $r>0.3$.

%These results show a broadening of the \Dptr\ distribution for low \pt\ particles inside the jet
%in central \pbpb\ collisions compared to those in \pp\ collisions while for higher \pt\ particles
%angular distributions are narrower in \pbpb\ collisions compared to \pp\ collisions.
%For all charged-particle \pt\ values the \RDptr\ values are greater than or equal to unity for
%$\rvar <$~0.05.
%Between $0.1 < r < 0.25$, a statistically significant trend of increasing \RDptr\ with increasing \ptjet\ is observed for low-\pt\ particles.
%No significant \ptjet\ dependence is seen for particles  with $\pt >$~4~\GeV.
%These new measurements will help distinguish modification of the jet, which is expected to be peaked near the jet core and 
%any response of the hot QCD matter to the jet.
%
