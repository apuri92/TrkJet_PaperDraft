% !TEX root = trkjet.tex

This paper presents a measurement of the yields of charged particle distributions, \Dptr, inside and around \RFour\ \antikt\ jets with $|\yjet| <$1.7 up to a distance of $r = 0.8$ from the jet axis. The yields are measured in intervals of \ptjet\ from 126 to 316~\GeV\ in \PbPb\ and \pp\ collisions at 5.02~\TeV\ as a function of charged particle \pt\ and the angular distance \rvar\ between the jet axis and charged particle.

Centrality dependent modifications to the yields, when compared to those measured in \pp\ collisions, are observed. The magnitude of these modifications increases with increasing collision centrality. 
The \RDptr\ distributions for charged particles with $\pt <$~4~\GeV\ 
are above unity and 
grow with increasing angular separation up to $r \sim0.3$, showing weak to no dependence on $r$ in the interval 0.3~$< \rvar <$~0.6 followed with a small decrease in the enhancement for 0.6~$< \rvar <$~0.8.
For charged particles with $\pt >$~4~\GeV, a suppression in \RDptr\ is observed, and the 
distributions decrease with increasing
\rvar\ for $\rvar < $~0.3, with no \rvar\ dependence for $r>0.3$. 
These results show a broadening of the \Dptr\ distribution for low \pt\ particles inside the jet
in central \pbpb\ collisions compared to those in \pp\ collisions while for higher \pt\ particles
angular distributions are narrower in \pbpb\ collisions compared to \pp\ collisions.
For all charged-particle \pt\ values the \RDptr\ values are greater than or equal to unity for
small \rvar\ values (inside the core of the jet).
Between $0.1 < r < 0.25$, a statistically significant trend of increasing \RDptr\ with increasing \ptjet\ is observed for low-\pt\ particles. No significant \ptjet\ dependence is seen for particles  with $\pt >$~4~\GeV.

These measurements provide insight into the differential distributions of charged particles within jets as compared to the inclusive measurement of jet fragmentation functions. % and are important
%for understanding how jets interact with the QGP.
They provide new information about our understanding of the physics of soft
gluon radiation and the response of the QGP to jets.
