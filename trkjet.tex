%-------------------------------------------------------------------------------
% This file provides a skeleton ATLAS paper.
%-------------------------------------------------------------------------------
% \pdfoutput=1
% The \pdfoutput command is needed by arXiv/JHEP/JINST to ensure use of pdflatex.
% It should be included in the first 5 lines of the file.
% \pdfinclusioncopyfonts=1
% This command may be needed in order to get \ell in PDF plots to appear. Found in
% https://tex.stackexchange.com/questions/322010/pdflatex-glyph-undefined-symbols-disappear-from-included-pdf
%-------------------------------------------------------------------------------
% Specify where ATLAS LaTeX style files can be found.
\newcommand*{\ATLASLATEXPATH}{latex/}
% Use this variant if the files are in a central location, e.g. $HOME/texmf.
% \newcommand*{\ATLASLATEXPATH}{}
%-------------------------------------------------------------------------------
\documentclass[PAPER, atlasdraft=true, texlive=2016, UKenglish, coverpage, PAPER]{\ATLASLATEXPATH atlasdoc}
% The language of the document must be set: usually UKenglish or USenglish.
% british and american also work!
% Commonly used options:
%  atlasdraft=true|false This document is an ATLAS draft.
%  texlive=YYYY          Specify TeX Live version (2016 is default).
%  coverpage             Create ATLAS draft cover page for collaboration circulation.
%                        See atlas-draft-cover.tex for a list of variables that should be defined.
%  cernpreprint          Create front page for a CERN preprint.
%                        See atlas-preprint-cover.tex for a list of variables that should be defined.
%  NOTE                  The document is an ATLAS note (draft).
%  PAPER                 The document is an ATLAS paper (draft).
%  CONF                  The document is a CONF note (draft).
%  PUB                   The document is a PUB note (draft).
%  BOOK                  The document is of book form, like an LOI or TDR (draft)
%  txfonts=true|false    Use txfonts rather than the default newtx
%  paper=a4|letter       Set paper size to A4 (default) or letter.

%-------------------------------------------------------------------------------
% Extra packages:
\usepackage[backend=bibtex]{\ATLASLATEXPATH atlaspackage}
% Commonly used options:
%  biblatex=true|false   Use biblatex (default) or bibtex for the bibliography.
%  backend=bibtex        Use the bibtex backend rather than biber.
%  subfigure|subfig|subcaption  to use one of these packages for figures in figures.
%  minimal               Minimal set of packages.
%  default               Standard set of packages.
%  full                  Full set of packages.
%-------------------------------------------------------------------------------
% Style file with biblatex options for ATLAS documents.
\usepackage{\ATLASLATEXPATH atlasbiblatex}

% Useful macros
\usepackage{\ATLASLATEXPATH atlasphysics}
\usepackage{multirow}
\input{macros}

% See doc/atlas_physics.pdf for a list of the defined symbols.
% Default options are:
%   true:  journal, misc, particle, unit, xref
%   false: BSM, heppparticle, hepprocess, hion, jetetmiss, math, process,
%          other, snippets, texmf
% See the package for details on the options.

% Files with references for use with biblatex.
% Note that biber gives an error if it finds empty bib files.
 \addbibresource{trkjet.bib}
\addbibresource{bib/ATLAS.bib}
\addbibresource{bib/CMS.bib}
\addbibresource{bib/ConfNotes.bib}
\addbibresource{bib/PubNotes.bib}

% Paths for figures - do not forget the / at the end of the directory name.
\graphicspath{{logos/}{figures/}}

% Add you own definitions here (file trkjet-defs.sty).
\usepackage{trkjet-defs}
\usepackage[utf8]{inputenc}

%-------------------------------------------------------------------------------
% Generic document information
%-------------------------------------------------------------------------------

% Title, abstract and document 
% !TEX root = trkjet.tex
%-------------------------------------------------------------------------------
% This file contains the title, author and abstract.
% It also contains all relevant document numbers used by the different cover pages.
%-------------------------------------------------------------------------------

% Title
\AtlasTitle{Measurement of angular and momentum distributions of charged particles within and around jets in \pbpb\ and $pp$ collisions at $\sqrt{s_{\mathrm{NN}}}=$~5.02~\TeV\ with ATLAS at the LHC}

% Draft version:
% Should be 1.0 for the first circulation, and 2.0 for the second circulation.
% If given, adds draft version on front page, a 'DRAFT' box on top of each other page, 
% and line numbers.
% Comment or remove in final version.
\AtlasVersion{0.1}

% Abstract - % directly after { is important for correct indentation
\AtlasAbstract{%
Studies of the fragmentation of jets into charged particles in heavy-ion collisions can provide information
about
 the mechanism of jet quenching by the hot and dense QCD matter created in such collisions, the quark-gluon plasma. 
This paper presents a measurement of the angular distribution of charged particles around the jet axis in $\sqrt{s_{\mathrm{NN}}}=$~5.02~\TeV\
 Pb+Pb and \textit{pp} collisions, done using the ATLAS detector at the LHC. The measurement is performed for jets reconstructed with the \antikt\ algorithm with radius parameter $R = $~0.4, and is extended to a distance of $r= 0.8$ outside the jet cone. The charged particles in this analysis have transverse momenta in the \mbox{1 GeV--63 GeV} range while jets have a transverse momenta in the \mbox{126 GeV--316 GeV} range and are restricted to have an absolute value of the jet rapidity be less 1.7. Results are presented as a function of \pbpb\ collision centrality and distance from the jet axis for different jet and charged-particle transverse momenta ranges. There is an enhancement in the yields of charged particles with transverse momenta below 4 GeV in Pb+Pb collisions compared to $pp$ collisions. This enhancement increases for larger angular distances from the jet axis, reaches a peak at $r = 0.6$ and decreases thereafter. Charged particles with transverse momenta above 4 GeV show a depletion in Pb+Pb collisions compared to the $pp$ reference. An enhancement of charged particles of all transverse momenta is also observed in the jet core for distances up to $r = 0.05$ from the jet axis. 
}

% Author - this does not work with revtex (add it after \begin{document})
% This has to be commented out for TDR etc.
 \author{The ATLAS Collaboration}

% ATLAS reference code, to help ATLAS members to locate the paper
\AtlasRefCode{ANA-HION-2018-03-PAPER}

% CERN preprint number
% \PreprintIdNumber{CERN-EP-2019-XX}

% ATLAS date - arXiv submission; usually filled in by the Physics Office
 \AtlasDate{\today}

% ATLAS heading - heading at top of title page. Set for TDR etc.
% \AtlasHeading{ATLAS ABC TDR}

% arXiv identifier
% \arXivId{14XX.YYYY}

% HepData record
% \HepDataRecord{ZZZZZZZZ}

% Submission journal and final reference
 \AtlasJournal{Phys.\ Rev.\ C.}
% \AtlasJournalRef{\PLB 789 (2017) 123}
% \AtlasDOI{}

%-------------------------------------------------------------------------------
% The following information is needed for the cover page. The commands are only defined
% if you use the coverpage option in atlasdoc or use the atlascover package
%-------------------------------------------------------------------------------

% List of supporting notes  (leave as null \AtlasCoverSupportingNote{} if you want to skip this option)
 \AtlasCoverSupportingNote{Correlations between jets and charged particles in Pb+Pb Collisions at 5.02 TeV}{https://cds.cern.ch/record/2304504}
% \AtlasCoverSupportingNote{Short title note 2}{https://cds.cern.ch/record/YYYYYYY}
%
% OR (the 2nd option is deprecated, especially for CONF and PUB notes)
%
% Supporting material TWiki page  (leave as null \AtlasCoverTwikiURL{} if you want to skip this option)
% \AtlasCoverTwikiURL{https://twiki.cern.ch/twiki/bin/view/Atlas/WebHome}

% Comment deadline
% \AtlasCoverCommentsDeadline{DD Month 2019}

% Analysis team members - contact editors should no longer be specified
% as there is a generic email list name for the editors
 \AtlasCoverAnalysisTeam{Akshat Puri, Anne Sickles, Martin Rybar}

% Editorial Board Members - indicate the Chair by a (chair) after his/her name
% Give either all members at once (then they appear on one line), or separately
 \AtlasCoverEdBoardMember{Mario Martinez Perez(chair), Iwona Grabowska-Bold, Benjamin Nachman}
% \AtlasCoverEdBoardMember{EdBoard~Chair~(chair)}
% \AtlasCoverEdBoardMember{EB~Member~1}
% \AtlasCoverEdBoardMember{EB~Member~2}
% \AtlasCoverEdBoardMember{EB~Member~3}

% A PUB note has readers and not an EdBoard--give their names here (one line or several entries)
% \AtlasCoverReaderMember{Reader~1, Reader~2}
% \AtlasCoverReaderMember{Reader~1}
% \AtlasCoverEdBoardMember{Reader~2}

% Editors egroup
 \AtlasCoverEgroupEditors{atlas-ana-hion-2018-03-analysis-team@cern.ch}

% EdBoard egroup
 \AtlasCoverEgroupEdBoard{atlas-ana-hion-2018-03-editorial-board@cern.ch}

% Author and title for the PDF file
\hypersetup{pdftitle={ATLAS document},pdfauthor={The ATLAS Collaboration}}

%-------------------------------------------------------------------------------
% Content
%-------------------------------------------------------------------------------
\begin{document}

\maketitle

%\tableofcontents

%-------------------------------------------------------------------------------
\section{Introduction}
\label{sec:intro}
% !TEX root = trkjet.tex

Ultra-relativistic nuclear collisions at the Large Hadron Collider (LHC) produce hot, dense matter called the quark-gluon plasma, QGP (see Refs.~\cite{Roland:2014jsa,Busza:2018rrf} for recent reviews).
Jets from hard-scattering processes in these collisions traverse and interact with the QGP, losing energy via a process called jet-quenching.
The rates and characteristics of these jets in heavy-ion collisions can be compared to the same quantities in \pp\ collisions, where we do not expect the production of QGP.
This comparison can provide information on the properties of the QGP and how it interacts with partons from the hard scatter.

Jets with large transverse momenta in central lead-lead (\pbpb) collisions at the LHC are measured at approximately half the rates in \pp\ collisions when the nuclear overlap function of \pbpb\ collisions is taken into account~\cite{Abelev:2013kqa,Aad:2014bxa,Adam:2015ewa,Khachatryan:2016jfl, 2019108}.
Similarly, back-to-back dijet~\cite{Aad:2010bu,Chatrchyan:2011sx,Aaboud:2017eww} and photon-jet pairs~\cite{Chatrchyan:2012gt,Aaboud:2018anc} are observed to have less balanced transverse momenta in \pbpb\ collisions compared to \pp\ collisions.
These observations suggest that some of the energy from the hard-scattered parton may be transferred outside of the jet through its interaction with the QGP medium.
 
Complementary measurements look at how the structure of jets is different between \pbpb\ and \pp\ collisions.
Jet shape measurements in the \pp\ and \pbpb\ collision systems have shown a broadening of the jets due to the QGP~\cite{Aad:2011sc, Acharya:2018uvf, Chatrchyan:2012mec, Chatrchyan:2013kwa}.
Additionally, measurements of jet fragmentation functions at the LHC show an excess, in PbPb collisions, of low and high momentum particles with a depletion of intermediate momentum particles inside the jet compared to pp collisions~\cite{Aad:2014wha,Chatrchyan:2014ava,Aaboud:2017bzv,Aaboud:2018hpb}.
Particles carrying a large fraction of the jet momentum are generally closely aligned with the jet axis, whereas low momentum particles are observed to have a much broader angular distribution extending outside the jet~\cite{Chatrchyan:2011sx,Khachatryan:2015lha,Khachatryan:2016tfj,Sirunyan:2018jqr}.
These observations suggest that the energy lost via jet-quenching is being transferred to soft particles around the jet axis via soft gluon emission~\cite{Vitev:2008rz,Ovanesyan:2011xy,Blaizot:2014ula,Qin:2015srf,Escobedo:2016jbm,Casalderrey-Solana:2016jvj,Tachibana:2017syd}.
Measurements of yields of these particles as a function of transverse momentum and angular distance between the particle and the jet axis have a potential to provide further insight into on the structure of jets in the QGP, as well as provide information on how the medium is affected by the presence of the jet.


This paper presents charged-particle \pt\ distributions around the jet axis that have been corrected for detector effects.
The measured yields are defined as:

\begin{align*}
\Dptr = \frac{1}{N_{\mathrm{jet}}} \frac{1}{A} \frac{\mathrm{d} n_{\mathrm{ch}} (\pt, r)}{\mathrm{d} \pt},
\end{align*}
where $r = \sqrt{\Delta \eta^2 + \Delta \phi^2}$ \footnote{ATLAS uses a right-handed coordinate system with its origin at the nominal interaction point (IP) in the centre of the detector, and the $z$-axis along the beam pipe.
The $x$-axis points from the IP to the centre of the LHC ring, and the $y$-axis points upward.
Cylindrical coordinates $(r,\phi)$ are used in the transverse plane, $\phi$ being the azimuthal angle around the $z$-axis.
The pseudorapidity is defined in terms of the polar angle $\theta$ as $\eta=-\ln\tan(\theta/2)$.
The rapidity is defined as $y = 0.5\text{ln}[(E + p_z)/(E-p_z)]$ where $E$ and $p_z$ are the energy and $z$-component of the momentum along the beam direction respectively.
Transverse momentum and transverse energy are defined as $\pt = p \sin\theta$ and $\Et = E \sin\theta$, respectively.
The angular distance between two objects with relative differences $\Delta \eta$ and $\Delta \phi$ in pseudorapidity and azimuth respectively is given by $\sqrt{(\Delta \eta )^2 + (\Delta \phi)^2}$.} 
is the angular distance from the jet axis and $N_{\mathrm{jet}}$ is the number of jets in consideration.
$A = \pi (r_{\mathrm{max}}^2 - r_{\mathrm{min}}^2) $ is the area of an annulus around the jet axis with its inner and outer radii $r_{\mathrm{min}}$ and $r_{\mathrm{max}}$ respectively and $n_{\mathrm{ch}}(\pt, r)$ is the number of charged particles with a given \pt\ within the annulus.
The ratios of the charged-particle yields measured in \pbpb\ and \pp\ collisions,

\begin{align*}
   \RDptr = \frac{\Dptr_\mathrm{Pb+Pb}}{\Dptr_{\pp}},
\end{align*}
quantify the modifications of the yields due to the QGP medium.
Furthermore, the differences between the \Dptr\ distributions in \pbpb\ and \pp\ collisions, 

\begin{align*}
   \Delta \Dptr = \Dptr_\mathrm{Pb+Pb} - \Dptr_{pp},
\end{align*}
allow for measuring the absolute differences in charged-particle yields between the two collision systems.




%-------------------------------------------------------------------------------
%-------------------------------------------------------------------------------
\section{ATLAS Detector}
\label{sec:setup}

% !TEX root = trkjet.tex

The measurements presented here are performed using the ATLAS calorimeter, inner detector, trigger, and data acquisition systems.
%The primary components used in the measurement of \Dptr\ distributions are the calorimeters and the inner detector.

The calorimeter system consists of a sampling liquid-argon (LAr) electromagnetic (EM) calorimeter covering $|\eta|<3.2$, a steel--scintillator sampling hadronic calorimeter covering $|\eta| <1.7$, LAr hadronic calorimeters covering $1.5 < |\eta| < 3.2$, and two LAr forward calorimeters~(FCal) covering $3.1 < |\eta| < 4.9$.

% and has a $\Delta
%\eta \times \Delta \phi$ granularity of $0.1 \times
%0.1$ for $|\eta| < 2.5$ and  $0.2 \times 0.2$ for $2.5 < |\eta| < 4.9$.\footnote{An 
%  exception is the third sampling layer that has a segmentation of $0.2 \times 0.1$
%up to $|\eta| = 1.4$.}

The EM calorimeters are segmented longitudinally in shower depth into three layers with an additional pre-sampler layer.
They have segmentation that varies with layer and pseudorapidity.
The hadronic calorimeters have three sampling layers longitudinal in shower depth \cite{Aad:2008zzm}.

%The EM calorimeters have a granularity that varies with layer and pseudorapidity, but which is generally much finer than that of the hadronic calorimeter.

The inner detector measures charged particles within the pseudorapidity interval $|\eta|<2.5$ using a combination of silicon pixel detectors, silicon microstrip detectors (SCT), and a straw-tube transition radiation tracker (TRT), all immersed in a 2~T axial magnetic field~\cite{Aad:2008zzm}.
Each of the three detectors is composed of a barrel and two symmetric end-cap sections.
The pixel detector is composed of four layers including the Insertable B-Layer~\cite{ibl1,ibl2}.
The SCT barrel section contains four layers of modules with sensors on both sides, and each end-cap consists of nine layers of double-sided modules with radial strips.
The TRT contains layers of staggered straws interleaved with the transition radiation material.


%The inner detector measures charged particles  within the pseudorapidity interval  $|\eta|<2.5$ using a combination of silicon pixel detectors, silicon microstrip detectors (SCT), and a straw-tube transition radiation tracker (TRT), all immersed in a 2~T axial magnetic field~\cite{Aad:2008zzm}.
%Each of the three detectors is composed of a barrel and two symmetric end-cap sections.
%The pixel detector  is composed of four layers: the "insertable B-layer" (IBL)~\cite{ibl1,ibl2} and three layers  with a pixel size of $50~{ \mu \mathrm{m}} \, \times \, 400~{ \mu \mathrm{m}}$.
% The  SCT barrel section contains four layers of modules with 80~$\mu {\mathrm m}$  pitch sensors on both sides and each end-cap consists of nine layers of double-sided modules with radial strips having a mean pitch of $80~\mathrm{\mu m}$.
%The two sides of each SCT layer in both the barrel and the end-caps have a relative stereo angle of 40~mrad.
%The TRT contains up to 73 (160) layers of staggered straws interleaved with fibres in the barrel (end-cap).

The zero-degree calorimeters (ZDCs) are located symmetrically at $z = \pm140$~m and cover $|\eta| > 8.3$.
The ZDCs use tungsten plates as absorbers, and quartz rods sandwiched between the tungsten plates as the active medium.
In \PbPb\ collisions the ZDCs primarily measure ``spectator'' neutrons.
These are neutrons that do not interact hadronically when the incident nuclei collide.
A ZDC coincidence trigger is implemented by requiring the pulse height from both ZDCs to be above a threshold that accepts the signal corresponding to the energy deposition from a single neutron.

This analysis uses the same trigger setup used in Ref.~\cite{Aaboud:2018hpb}, and is briefly described below.
A two-level trigger system is used to select the \PbPb\ and \pp\ collisions.
The first level is based on custom electronics while the second level, the High Level Trigger (HLT), is based on software \cite{Aaboud:2016leb}.
Minimum-bias~(MB) events are recorded using a logical OR of two triggers: 1) total energy Level-1 trigger; 2) ZDC coincidence trigger at Level-1 and a veto on the total energy trigger, with the additional requirement of least one track in the HLT.
The total-energy trigger requires a total transverse energy measured in the calorimeter system to be greater than 50~\GeV.
Jet events are selected by the HLT, seeded by a jet identified by the Level-1 jet trigger in \pp\ collisions or by the total-energy trigger with a threshold of 50~\GeV\ in \PbPb\ collisions.
The Level-1 jet trigger utilized in \pp\ collisions requires a jet with transverse momentum greater than 20~\GeV.
The HLT jet trigger uses a jet reconstruction procedure similar to that in the offline analysis as discussed in Section~\ref{sec:reconstruction}.
It selects events containing jets with a transverse energy of at least 75~\GeV\ in \PbPb\ collisions and at least 85~\GeV\ in \pp\ collisions.
The measurement is performed in the jet transverse momentum range where the trigger is fully efficient.

%the minimum-bias trigger, with two total transverse-energy triggers requiring 1.5~\TeV\ and 6.5~\TeV\ being used to enhance the rate of more central \PbPb\ events.





%-------------------------------------------------------------------------------
\section{Data sets and event selection}
\label{sec:data}
% !TEX root = trkjet.tex

The \PbPb\ and \pp\ data used in this analysis were recorded in 2015.
 The data samples consist of 25~pb$^{-1}$ of \sqrts~=~5.02~\TeV\ \pp\ and 0.49~nb$^{-1}$ of \sqrtsnn~=~5.02~\TeV\
\pbpb\ data. In both samples, events are required to have a reconstructed vertex
within 150~mm of the nominal interaction point along the beam axis.
Only events taken during stable beam conditions and satisfying detector and data-quality requirements that include the calorimeters and inner tracking detectors being in nominal operating conditions are considered. 

%The MB
%\PbPb\ sample was recorded with different pre-scales\footnote{The pre-scale indicates which fraction of events that passed the trigger selection was selected for recording by the data  acquisition.}  depending on the instantaneous luminosity in the
%LHC fill. The MB trigger recorded an effective luminosity of 22 $\mu$b$^{-1}$.


In \PbPb\ collisions, the event centrality reflects the overlap area of the two colliding nuclei and is characterized by \ETfcal, the total transverse energy deposited in the 
FCal~\cite{Aaboud:2017tql}. The six centrality intervals used in this analysis are defined according to successive percentiles of the \ETfcal\ distribution obtained in minimum-bias collisions, ordered from the most central (highest \ETfcal) to the most peripheral (lowest \ETfcal) collisions: 0--10\%, 10--20\%, 20--30\%, 30--40\%, 40--60\%, 60--80\%. 

In addition to the jet-triggered sample, a separate minimum bias \PbPb\ data sample was recorded with three trigger selections: the minimum-bias trigger and two total transverse-energy triggers with thresholds of 1.5 TeV and 6.5 TeV used to enhance the rate of central \pbpb\ events. This sample is combined with a set of $1.8\times10^7$ 5.02 TeV hard-scattering dijet \pp\ events generated with \powheg{}+\pythiaeight\ \cite{Nason:2004rx,Sjostrand:2014zea} using the A14 tune of parameters \cite{ATLAS2014021} and the NNPDF23LO PDF set \cite{Ball:2012cx} to produce the Monte Carlo Overlay sample. This sample is reweighted on an event by event basis such it has the same centrality distribution as the jet triggered sample.
A separate set of $1.8\times10^7$ 5.02 TeV hard-scattering dijet \pp\ events generated with the same tune and PDFs was used as the \pp\ MC. The detector response is simulated in both MC samples using \textsc{Geant4} \cite{Agostinelli:2002hh,Aad:2010ah} and they are then used to evaluate the performance of the detector and analysis procedure. Another sample of minimum bias \pbpb\ events is generated using HIJING (version 1.38b) \cite{Aad:2010ah} and is used to evaluate the track reconstruction performance.


%A sample of 1.8~$\times10^{7}$ simulated 5.02~\TeV\ \powheg{}+\pythiaeight~\cite{Nason:2004rx,Sjostrand:2014zea} \pp\ hard-scattering events, generated using the A14 tune~\cite{ATLAS2014021} and the NNPDF23LO PDF set~\cite{Ball:2012cx}, is used to evaluate the performance for measuring \Dptr\ distributions in the \pp\ data. The performance of the detector and analysis procedure in \PbPb\ collisions is evaluated using 1.8~$\times10^{7}$ 5.02~\TeV\ hard-scattering dijet events generated with \powheg{}+\pythiaeight\ overlaid on top of events from the enhanced minimum-bias \PbPb\ data sample. In both samples, the detector response is simulated using \textsc{Geant}4~\cite{Agostinelli:2002hh,Aad:2010ah}. A weight is assigned to each MC event such that the event sample obtained from the simulation has the same \ETfcal\ distribution as in jet triggered data.



%-------------------------------------------------------------------------------
\section{Jet and track selection}
\label{sec:reconstruction}
% !TEX root = trkjet.tex

The jet reconstruction procedure is identical to that used in Ref.~\cite{2019108}. 
%closely follow those used by \mbox{ATLAS} for jet measurements in \pp\ and \PbPb\ collisions at $\sqrtsnn=2.76$~\TeV~\cite{Aad:2014bxa} and 5.02 TeV \cite{2019108}.
The \antikt\ algorithm is first run in four-momentum recombination mode, on
$\Delta \eta \times \Delta \phi = 0.1\times 0.1$  calorimeter towers with two \antikt\ distance parameter values ($R=$~0.2 and $R=$~0.4). The energies in the towers are obtained by summing the
	energies of calorimeter cells at the electromagnetic energy scale within the tower boundaries. Then,
	  an iterative procedure is used to estimate the $\eta$-dependent underlying event (UE)  transverse energy density, while excluding the regions populated by jets. The estimate of the UE contribution is performed on an event-by-event basis.
	Furthermore, the background is modulated to account for the azimuthal anisotropy in particle production~\cite{ATLAS:2012at}. The modulation accounts for the contribution of the second, third, and fourth order azimuthal anisotropy harmonics.
	The UE transverse energy is subtracted from calorimeter towers included in the jet and the four-momentum of the jet is updated accordingly.
	  Then, a jet $\eta$- and \pT-dependent  correction factor to the \ptjet\ 
	  derived from the simulation samples is applied to correct for the calorimeter energy
	  response~\cite{Aaboud:2017jcu}. The same calibration factors are applied both 
in \pp\ and \pbpb\ collisions.
An additional correction based on \textit{in-situ} studies of jets recoiling against photons, $Z$ bosons, and jets in other regions of the calorimeter is
	  applied~\cite{ATL-PHYS-PUB-2015-036,2019167}. The same jet reconstruction procedure without the
	  azimuthal modulation of the UE is also applied to $pp$ collisions.
	  In this analysis, jets are required to have \ptjet\ in the 126--316 \GeV\ range, with rapidity  $|\yjet|<$~1.7. The \ptjet\ cut was chosen based on the fake rate below \mbox{100 GeV}, while the rapidity cut was based on the acceptance of the tracking system.
 To prevent nearby jets from distorting the measurement of \Dptr\ distributions, 
jets are rejected if there is another jet with a higher \ptjet\ than the considered jet anywhere
within an angular distance of $\Delta R < 1.0$. The isolation requirement removes approximately 0.01\% of jets.

Charged-particle tracks in \pbpb\ collisions are reconstructed from hits in the inner detector using the 
track reconstruction algorithm that has been optimized for the high hit density in heavy-ion
collisions~\cite{Aaboud:2017all}.
Tracks used in this analysis have $|\eta| < 2.5$ and are required to have at least 9 (11) total silicon hits for charged particles with pseudorapidity,  \mbox{|\etatrk| $\leq$ 1.65 (|\etatrk| > 1.65)}.  At least one hit is required in one of the two innermost pixel layers.
If the track trajectory passes through an active module in the innermost layer, then 
a hit in this layer is required. Additionally, a track must 
have no more than two holes in the pixel and SCT detectors together, where 
a hole is defined by the absence of a hit predicted by the track 
trajectory. 
All charged-particle tracks used in this analysis are required to have reconstructed transverse momentum $\pttrk > 1.0 $~\GeV. In order to suppress a contribution from
secondary particles\footnote{Primary particles are defined as particles with a mean lifetime $\tau>0.3\times 10^{-10}$s either directly produced in \pp\ interactions or from subsequent decays of particles with a shorter lifetime. All other particles are considered to be secondary.}, the distance of closest approach of the track to the primary vertex is required to be less than a value that varies from 0.45~mm at $\pttrk=4$~\GeV\ to 0.2~mm at $\pttrk=20$~\GeV\ in the transverse plane and is less than 1.0~mm in the longitudinal direction.


The efficiency, $\varepsilon$, for reconstructing charged particles in \PbPb\ and \pp\ collisions is determined using the MC samples described above. It is evaluated as a function of the generator-level primary particle transverse momentum, \pTtrue, and pseudorapidity, \etatrue, by associating tracks to generator-level primary particles~\cite{Aad:2010ah}. For \pbpb\ collisions the efficiency is also evaluated separately in each centrality interval used in the measurement. In both collision systems the efficiency increases slowly with \pTtrue\ and is seen to be independent of \ptjet\ in the measurement phase space. It is approximately 80\% at \mbox{$\pt = 1$ GeV} and rises to approximately 90\% at \mbox{$\pt = 63$ GeV}. The variation in efficiency between the most central and peripheral \pbpb\ collisions is approximately 3\%.

The contribution of reconstructed tracks that cannot be matched to a generated primary particle in the \pp\ MC samples (this includes ``fake'' tracks, as well as tracks that are matched to secondary particles), are together less than 2\% in the entire \pttrk\ range under study in both \pp\ and \pbpb\ collisions.  









%-------------------------------------------------------------------------------
\section{Analysis procedure}
\label{sec:analysis}
% !TEX root = trkjet.tex

The analysis procedure is similar to that in Ref.~\cite{Aaboud:2018hpb} with the additional requirement of being done differentially in \rvar. Measured tracks are associated with a reconstructed jet if they fall within $\Delta R < 0.8$ of the jet axis and are constructed as:

\begin{align*}
\dfrac{\fd^{2} \nchmeas }{ \fd \pttrk \fd r} = \frac{1}{\varepsilon(\pttrk,\etatrk)} \frac{\Delta \Nch( \pttrk, r)}{ \Delta\pttrk \Delta r}
\end{align*}

where $\Delta \Nch (\pttrk, r)$ represents the number of tracks within a given \pttrk\ and $r$ range. The efficiency correction is applied as a $1/\varepsilon(\pttrk,\etatrk)$ weight on a track-by-track basis, assuming $\pttrk = \pTtrue$. While that assumption is not strictly valid, the efficiency varies sufficiently slowly with $\pTtrue$ that the error
introduced by this assumption is less than 1\%. It is further corrected for by the Bayesian unfolding procedure described later in this section.

The measured track yields need to be corrected for the UE, fake tracks and secondaries. In \pp\ collisions, the UE contribution from hard scatterings not associated with jet production is negligible. The contributions from fake tracks and secondary charged particles are estimated from MC samples and subtracted. This procedure is similar to that applied in previous measurements~\cite{Aaboud:2017tke,Aaboud:2018hpb}.

For \pbpb\ collisions, the UE, fake track, and secondary contributions are estimated together in a two step process: first, MC overlay is used to generate \etajet--\phijet\ maps of the average number of charged particles in a given annulus around a reconstructed jet. This is done for charged particles without a truth match and as a function of \ptjet, \etajet, \phijet, angle of the jet to the second order event plane\footnote{The second order event plane angle $\Psi_2$ is determined on an event-by-event basis by a standard method using the $\phi$ variation of transverse energy in the FCal \cite{ATLAS:2012at}} $ \mathrm{d}\Psi_{\mathrm{jet}}$, \rvar, \pttrk, and centrality. In the second step, the \etajet--\phijet\ maps are used to generate the UE distribution for jets with a given \etajet, \phijet, and $\mathrm{d}\Psi_{\mathrm{jet}}$. This distribution includes fakes, and is given by \mbox{$\fd^2 \nch^{\mathrm{UE+Fake}}(\pttrk, r) / \fd \pttrk \fd r$}. The yields decrease with decreasing collision centrality, increasing \pttrk, and increasing azimuthal distance from $\Psi_2$. The subtracted distributions are then given by 

\begin{align*}
\frac{\fd^2 \nchsub }{ \fd \pttrk \fd r } &=  \frac{\fd^2 \nchmeas }{ \fd \pttrk \fd r} -  \frac{ \fd^2 \nch^{\mathrm{UE+Fake}}(r)  }{ \fd \pttrk \fd r} 
\end{align*}

Figure~\ref{fig:UEsize} shows the ratio of the charged-particle distributions before and after the subtraction of the UE, fake tracks, and secondaries,
%the charged-particle distributions prior to the UE and fake track subtraction, $ \fd^2 \nchmeas / \fd \pttrk \fd r$, divided by the distributions after the subtraction, $ \fd^2 \nchsub / \fd \pttrk \fd r $
 as a function of \rvar\ for different \pttrk\ intervals and $126 < \ptjet < 158$ \GeV\ for six centrality selections. The largest UE contribution is for 1.0~\GeV\ charged particles at large values of \rvar\ in central collisions, with the background being approximately 100 times the signal, and slowly decreasing with increasing \ptjet. It rapidly decreases for more peripheral collisions, larger \pttrk\ and smaller \rvar. In addition, due to the steeply falling nature of the jet \pt\ spectra, the smearing due to jet energy resolution leads to a net migration of jets from lower \ptjet\ to higher \ptjet\ values, such that a jet reconstructed with a given transverse momentum will correspond, on average, to a lower truth jet \pT.  This ``up-feeding'' induces a difference between the UE yields determined using the MC overlay events and the actual UE contribution to reconstructed jets. The magnitude of this difference is centrality dependent, and can be seen clearly for charged particles with $\pt > 10$ GeV.

\begin{figure}
\centerline{
 \includegraphics[width=0.95\textwidth]{figures/performance/UE_B2S_single_0.pdf} }
\caption{ Ratio of the raw charged particle distributions to those after the subtraction of the UE and fake tracks as a function of \rvar\ for different \pttrk\ intervals, six centrality selections and for \ptjet\ between 126--158~\GeV.}
\label{fig:UEsize}
\end{figure}


%The fraction of fake tracks is found to be below 2\% of the tracks that pass the selection in all track and jet kinematic regions in this analysis.

To remove the effects of the bin migration due to the jet energy and track momentum resolution, the subtracted $\fd^2 \nchsub /\fd\pttrk \fd r$ distributions are corrected by a two-dimensional Bayesian unfolding~\cite{DAgostini:1994zf}
in \pttrk\ and \ptjet\ as implemented in the RooUnfold package~\cite{Adye:2011gm}.  
Two-dimensional unfolding is used because the calorimetric jet energy response depends on the fragmentation pattern of the jet~\cite{Aad:2011he}.
Four-dimensional response matrices are created from the \pp\ and \pbpb\ MC samples using the generator-level and reconstructed \ptjet, and the generator-level and reconstructed charged-particle \pttrk. They are corrected for tracking efficiencies and are evaluated in bins of \rvar\ and centrality. The Bayesian procedure requires a choice in the number of iterations.
Additional iterations reduce the sensitivity to the choice of prior, but may
amplify statistical fluctuations in the distributions.
After four iterations the 
charged particle distributions are found to be stable for both the \PbPb\ and \pp\ data.
A separate one-dimensional Bayesian unfolding is used to correct the measured \ptjet\ spectra that are used to normalize the unfolded charged particle distributions.
To achieve better correspondence with the data, the response matrices for both the one and two dimensional unfolding are reweighed so that the charged particle and jet distributions match the shapes in the reconstructed data.

An independent bin-by-bin unfolding procedure is also used to correct for migrations originating from the finite jet and track angular resolutions. Two corresponding \Dptr\ distributions are evaluated in MC samples, one using truth jets\footnote{Truth jets are reconstructed by applying the \antikt\ algorithm to stable final-state particles from MC generators like \PYTHIA. Particles are required to have a lifetime of $c\tau > 10$ mm and muons, neutrinos, and particles from pile-up activity are excluded} and primary particles and the other using reconstructed jets and charged particles with their reconstructed \pt\ replaced by generator-level transverse momentum, \pTtrue. The ratio of these two MC distributions provides a correction factor which is then applied to the data. 

The final particle-level corrected distributions, normalized by the area of the annulus under question are defined as:

\begin{align*}
   \Dptr = \frac{1}{N_\mathrm{jet}^\mathrm{unfolded}} \frac{1}{A(r)} \frac{\fd^2 \nchunf(r)}{\fd \pt \fd r},
 \end{align*}
where $N_\mathrm{jet}^\mathrm{unfolded}$ is the unfolded number of jets in a given \ptjet\ interval, and \nchunf\  is the unfolded yield of charged particles with a given \pt in an annulus of area $A$ at a distance \rvar.

The performance of the full analysis procedure is validated in the MC samples by comparing the fully corrected charged particle distributions to the generator-level distributions. Good closure (< 4 \%) is seen for charged particles with $\pt < 10$ GeV in both the \pp\ and \pbpb\ collision systems. The non-closure is taken as an additional systematic uncertainty as discussed in Section~\ref{sec:systematics}.
It is to be noted that adding or removing particles carrying a large fraction of the jet momentum near the edge of the jet can significantly alter its
reconstructed momentum and position; this instability leads to some non-closure in the analysis procedure for particles with $\pt > 10$ GeV in jets with $\ptjet < 200$ GeV. 
Results are presented where the non-closure in the \pp\ MC sample is less than 5\%.

%, excluding the following regions of phase space: 6--10 GeV tracks above $\rvar > 0.3$, 10--25 GeV tracks above $\rvar > 0.3$, and 25--63 GeV tracks above $\rvar > 0.2$ for 126--158 GeV jets; 10--25 GeV tracks above $\rvar > 0.4$, and 25--63 GeV tracks above $\rvar > 0.3$ for 158--200 GeV jets; 25--63 GeV tracks above $\rvar > 0.3$ for 200--251 GeV jets.




%-------------------------------------------------------------------------------
\section{Systematic uncertainties}
\label{sec:systematics}
% !TEX root = trkjet.tex

The following sources of systematic uncertainty are considered:
the jet energy scale (JES), the jet energy resolution (JER), 
the sensitivity of the  unfolding to the prior, the UE contribution, the residual non-closure of the analysis procedure, and tracking-related uncertainties.
For each systematic variation, the \Dptr\ distributions along with their ratios and differences are re-evaluated. The difference between the varied and nominal distributions is used as an estimate of the uncertainty.

The systematic uncertainty due to the JES in \PbPb\ collisions is composed of two parts: 
a centrality-independent baseline component and a centrality-dependent component. Only the centrality-independent baseline component is used in \pp\ collisions; 
it is determined from \textit{in-situ} studies of the calorimeter
response~\cite{Aad:2011he,HIjesnote,Aaboud:2017jcu} and the relative energy scale difference between the jet reconstruction procedure in heavy-ion collisions~\cite{HIjesnote} and the procedure used in \pp\ collisions~\cite{Aad:2014bia}. The centrality-dependent uncertainty reflects a modification of parton showers by the \PbPb\ environment. It is evaluated by comparing calorimeter \ptjet\ and the sum of the transverse momentum of charged particles within the jet in data and MC. The size of the centrality-dependent uncertainty on the JES reaches 0.5\% in the most central collisions. Each component that contributes to the JES uncertainty is varied separately by $\pm1$ standard deviation for each interval in \ptjet\ and the response matrix is recomputed accordingly. The data are unfolded with modified matrices. The resulting uncertainty from the JES increases with increasing charged-particle \pT\ at fixed \ptjet\ and decreases with increasing \ptjet, and is at the level of 2--4\%.

The uncertainty on the \Dptr\ distributions due to the JER is evaluated by repeating the unfolding procedure with modified response matrices, where an additional contribution is added to the resolution of the reconstructed \ptjet\ using a Gaussian smearing procedure. The smearing factor is evaluated using an \textit{in-situ} technique in 13~\TeV\ \pp\ data involving studies of dijet energy balance~~\cite{Aad:2012ag,JERConfNote}. An additional uncertainty is included to account for differences between the tower-based jet reconstruction and that used in analyses of 13~\TeV\ \pp\ data. The resulting uncertainty from the JER is symmetrized to account for negative variations of the JER.  The size of the resulting uncertainty on the \Dptr\ distributions due to the JER typically reaches 4--5\% for the highest charged-particle \pT\ intervals and decreases to 2--3\% with decreasing charged-particle \pT\ at fixed \ptjet.


The uncertainties related to track reconstruction and selection originate from several sources.
Uncertainties related to the material description in simulation and the track transverse 
momentum resolution are obtained from studies in data and simulation described in Ref.~\cite{ATL-PHYS-PUB-2015-051}.
The sensitivity of the tracking efficiency to the description of the 
inactive material in the MC samples is evaluated by varying the material description.
This resulting uncertainty in the track reconstruction efficiency is between
0.5\% and 2\% in the track \pT\ range used in the analysis. 
The systematic uncertainty on the fake track rate is 30\% in both collision systems~\cite{ATL-PHYS-PUB-2015-051}.  The contamination of fake tracks is less than 2\% and the resulting uncertainty in the \Dptr\ distributions is at most 5\%.
An additional uncertainty takes into account a possible residual misalignment of the tracking detectors
in \pp\ and \PbPb\ data-taking. The alignment in these datasets is checked \textit{in-situ} with $Z\rightarrow \mu^{+}\mu^{-}$ events, and thus a track-\pT\-dependent uncertainty arises from the finite size of this sample. The resulting uncertainties in
the \Dptr\ distributions are typically less than 0.1\%. An additional  uncertainty in the tracking efficiency due to the high local track density in the core of jets is 0.4\%~\cite{ATL-PHYS-PUB-2016-007} for all \ptjet\ ranges in this analysis. The uncertainty due to the track selection is evaluated by repeating the analysis with an additional requirement on the significance of the distance of closest approach of the track to the primary vertex. This uncertainty affects 
the track reconstruction efficiencies, track momentum resolution, and rate of fake tracks. The resulting uncertainty typically varies between 1--2\%.
Finally, the track-to-particle association requirements are varied. This variation affects the track reconstruction efficiency, track momentum resolution, and rate of fake tracks. The resulting systematic uncertainty is $\leq~0.1 \%$ on the \Dptr\ distributions. All track-related systematic uncertainties are added in quadrature and presented as total tracking uncertainty. 

The systematic uncertainty associated with the UE subtraction has two components: limited statistics of charged particles associated with a jet without a corresponding generator particle in the \pbpb\ MC, and a comparison to an alternative UE estimation done using the cone method. The cone method uses jet triggered events to estimate the background and is adapted from \cite{Aaboud:2018hpb, Aaboud:2017bzv}. A regular grid of 9 cones of size $R = 0.8$ is used to cover the inner detector region. Cones are excluded if they are within an angular distance of 1.6 to a reconstructed jet with $\ptjet > 90$ GeV or if they contain a charged particle with \mbox{$\pt > 10$ GeV}. This exclusion reduces biases from any hard processes. The resulting UE charged particle yields $\fd \nchUE^{\mathrm{Cone}}/ \fd \pTch$ are evaluated over the \mbox{1--10 GeV} range as a function of \pttrk, \ptjet, centrality, and \rvar, and are subsequently averaged over all cones. The UE uncertainty on the \Dptr\ distributions is approximately 40\% at the largest angular distances from the jet axis and rapidly decreases with increasing charged-particle \pT\ and decreasing distance. This is the dominant source of the systematic uncertainty at low charged-particle \pt\ and large \rvar. In particular, the component from the limited statistics dominates in the most central collisions, while the component from the alternative estimation method dominates elsewhere.


The systematic uncertainty on the unfolding procedure is estimated by generating the response matrices from the MC distributions without any re-weighting to match shapes in data. Conservatively, the difference between the nominal \Dptr\ distribution and \Dptr\ unfolded with the re-weighted response matrices is taken as the systematic uncertainty, and is at the level of 5--7\%.

Conservatively, an additional uncertainty to account for possible residual limitations in the analysis procedure is assigned by evaluating the non-closure of the unfolded distributions in simulations. This is typically at the level of 3-4\% and is described in Section~\ref{sec:analysis}.

The correlations between the various systematic components are considered in evaluating the \RDptr\ and $\Delta\Dptr$ distributions. The unfolding and non-closure uncertainty are taken to be uncorrelated between \pp\ and \pbpb\ collisions, while all other uncertainties are taken to be correlated. For these, the \RDptr\ and $\Delta\Dptr$ distributions are re-evaluated by applying the variation to both collision systems; the resulting variations of the ratios from their central values are used as the correlated systematic uncertainty. 

Examples of systematic uncertainties in the \Dptr\ distributions for jets in the 126--158~\GeV\ \ptjet\ 
range measured in \pp\ and \pbpb\ collision systems are shown in Figure~\ref{fig:Systematics_Dpt}. The uncertainties on the \RDptr\ distributions are shown in Figure~\ref{fig:Systematics_RDpT}. It can be seen that the dominant systematic uncertainty on the \pbpb\ and the \RDptr\ distributions is from the underlying event estimation. While it is less than 5\% for $r < 0.3$ from the jet axis, it is approximately 40\% for charged particles with $\pt = 1$ GeV at $r = 0.8$ from the jet axis. The uncertainties in the \pp\ system are smaller, with the dominant systematic uncertainty coming from the tracking. This uncertainty is approximately 10\%  for $r < 0.1$ and decreases to less than 5\% at larger distances.

\begin{figure}
\centerline{
\begin{tabular}{cc}
\includegraphics[width=0.53\textwidth]{figures/systematics/ChPS_dR_sys_pp_error_trk2_jet7_cent6} &
\includegraphics[width=0.53\textwidth]{figures/systematics/ChPS_dR_sys_pp_error_trk6_jet7_cent6} \\
\includegraphics[width=0.53\textwidth]{figures/systematics/ChPS_dR_sys_PbPb_error_trk2_jet7_cent0} &
\includegraphics[width=0.53\textwidth]{figures/systematics/ChPS_dR_sys_PbPb_error_trk6_jet7_cent0} \\
\includegraphics[width=0.53\textwidth]{figures/systematics/ChPS_dR_sys_PbPb_error_trk2_jet7_cent5} &
\includegraphics[width=0.53\textwidth]{figures/systematics/ChPS_dR_sys_PbPb_error_trk6_jet7_cent5} \\
\end{tabular}}
\caption{
Relative size of the systematic uncertainties for \Dptr\ distributions in \pp\ (top), central 0--10\% \pbpb\ (middle), and peripheral 60--80\% \pbpb\ (bottom) collisions for tracks with $1.0 < \pt < 1.6$ \GeV\ (left) and $6.3 < \pt < 10$ \GeV\ (right) in jets with $126 < \ptjet < 158$ \GeV. The systematic uncertainties due to JES, JER, unfolding, UE contribution, MC non-closure, and tracking are shown along with the total systematic uncertainty from all sources.
}
\label{fig:Systematics_Dpt}
\end{figure}

\begin{figure}
\centerline{
\begin{tabular}{cc}
\includegraphics[width=0.53\textwidth]{figures/systematics/RDpT_dR_sys_error_trk2_jet7_cent0} &
\includegraphics[width=0.53\textwidth]{figures/systematics/RDpT_dR_sys_error_trk6_jet7_cent0} \\
\includegraphics[width=0.53\textwidth]{figures/systematics/RDpT_dR_sys_error_trk2_jet7_cent5} &
\includegraphics[width=0.53\textwidth]{figures/systematics/RDpT_dR_sys_error_trk6_jet7_cent5} \\
\end{tabular}}
\caption{
Relative size of the systematic uncertainties for \RDptr\ distributions for 0--10\% (top) and 60--80\% (bottom) \pbpb\ collisions, for tracks with $1.0 < \pt < 1.6$ \GeV\ (left) and $6.3 < \pt < 10.0$ \GeV\ (right), in jets with $126 < \ptjet < 158$ \GeV. The systematic uncertainties due to JES, JER, unfolding, UE contribution, MC non-closure, and tracking are shown along with the total systematic uncertainty from all sources.
}
\label{fig:Systematics_RDpT}
\end{figure}

%\begin{figure}
%\centerline{
%\begin{tabular}{cc}
%\includegraphics[width=0.55\textwidth]{figures/systematics/DeltaDpT_dR_sys_error_trk3_jet7_cent0} &
%\includegraphics[width=0.55\textwidth]{figures/systematics/DeltaDpT_dR_sys_error_trk6_jet7_cent0} \\
%\end{tabular}}
%\caption{
%Relative size of the systematic uncertainties for $\Delta\Dptr$ distributions for 0--10\% \pbpb\ collisions, for tracks in the \pt\ range 1.0--1.6 \GeV\ (left) and 6.3--10.0 \GeV\ (right), in jets with $126 < \ptjet < 158$ \GeV. The systematic uncertainties due to JES, JER, unfolding, UE contribution, MC non-closure and tracking are shown along with the total systematic uncertainty from all sources.
%}
%\label{fig:Systematics_DeltaDpT}
%\end{figure}


%-------------------------------------------------------------------------------
\section{Results}
\label{sec:results}
% !TEX root = trkjet.tex

The \Dptr\ distributions are studied as a function of \ptjet\ for \pp\ data and \PbPb\ collisions with different centralities.
The interplay between the hot and dense matter and the parton shower is explored by evaluating the ratios and differences between the \Dptr\ distributions in \pbpb\ and \pp\ collisions.
Some selected moments of these distributions are also investigated.



%%%%%%%    DPtr distributions    %%%%%%%
\subsection{\Dptr\ distributions}
\label{sec:dptr}
The \Dptr\ distributions evaluated in \pp\ and \pbpb\ collisions for $126 < \ptjet < 158$ GeV are shown in Figure~\ref{fig:dptr}.
These distributions decrease as a function of distance from the jet axis.
The rate at which they fall off sharply increases for higher \pt\ particles, with most of these being concentrated near the jet axis.
The distributions exhibit a difference in shape between \PbPb\ and \pp\ collisions, with the \pbpb\ distributions being broader at low \pt\ (\pt < 4 GeV) and narrower at high \pt\ (\pt > 4 GeV) in \mbox{0--10\%} central collisions.
This modification is centrality dependent and is smaller for peripheral \pbpb\ collisions.

\begin{figure}[h]
\centerline{
\begin{tabular}{ccc}
\includegraphics[width=0.36\textwidth]{figures/results/DpT_dR_jet7_cent0} &
\includegraphics[width=0.36\textwidth]{figures/results/DpT_dR_jet7_cent1} &
\includegraphics[width=0.36\textwidth]{figures/results/DpT_dR_jet7_cent2} \\
\includegraphics[width=0.36\textwidth]{figures/results/DpT_dR_jet7_cent3} &
\includegraphics[width=0.36\textwidth]{figures/results/DpT_dR_jet7_cent4} &
\includegraphics[width=0.36\textwidth]{figures/results/DpT_dR_jet7_cent5} \\
\end{tabular}}
\caption{The \Dptr\ distributions in \pp\ (open symbols) and \pbpb\ (closed symbols) as a function of angular distance $r$ for \ptjet\ of 126 to 158~\GeV.
The colors represent different track \pt\ ranges, and each panel is a different centrality selection.
The vertical bars on the data points indicate statistical uncertainties while the shaded boxes indicate systematic uncertainties.
The widths of the boxes are not indicative of the bin size and the points are shifted horizontally for better visibility.
The distributions for $\pt > 6.3$ GeV are restricted to smaller \rvar\ values as discussed in Section~\ref{sec:analysis}.}
\label{fig:dptr}
\end{figure}



%%%%%%%    RDptr distributions    %%%%%%%
\subsection{\RDptr\ distributions}
\label{sec:rdptr}
In order to quantify the differences seen in Figure~\ref{fig:dptr}, ratios of the \Dptr\ distributions in \pbpb\ collisions to those measured in \pp\ collisions for $126 < \ptjet < 158$ GeV and $200 < \ptjet < 251$ GeV jets are presented in Figure~\ref{fig:rdptr}.
They are shown as a function of $r$ for different \pt\ and centrality selections.
In 0--10\% central collisions, \RDptr\ is greater than unity for $\rvar < 0.8$ for charged particles with \pT less than 4.0~\GeV\ for both \ptjet\ selections.
For these particles, the enhancement of yields in \pbpb\ collisions compared to those in \pp\ collisions grows with increasing \rvar\ up to approximately \mbox{$\rvar  = 0.3$}, with \RDptr\ reaching up to two for 1.0~$< \pt <$~2.5~\GeV.
The value of \RDptr\ is approximately constant for \rvar\ in the interval \mbox{0.3--0.6} and decreases for \mbox{$\rvar > 0.6$}.
For charged particles with $\pt > 4.0$ \GeV, \RDptr\ shows a depletion outside the jet core for $r > 0.05$.
The magnitude of this depletion increases with increasing \rvar\ up to $r = 0.3$ and is approximately constant thereafter.
For 30--40\% mid-central collisions, the enhancement of particles with $\pt < 4.0$~\GeV\ has similar trends to that in the most central collisions, however the depletion of particles with $\pt > 4.0$~\GeV\ is not as strong.
For 60--80\% peripheral collisions, \RDptr\ has no significant \rvar\ dependence and the values of \RDptr\ are within approximately 50\% of unity.

The observed behavior inside the jet cone, $r < 0.4$, agrees with the measurement of the inclusive jet fragmentation functions~\cite{Aaboud:2017eww, Aaboud:2017bzv, Aaboud:2018hpb}, where yields of fragments with $\pt < 4$ GeV are observed to be enhanced and yields of charged particles with intermediate \pT\ are suppressed in \PbPb\ collisions compared to those in \pp\ collisions.
Calculations done in Ref.~\cite{Tachibana:2017syd} show that the medium response to the jet compensates the energy that is lost by the jet in \pbpb\ collisions even up to $r = 1.0$ from the jet axis.
The plateauing and slight decrease seen in Figure~\ref{fig:rdptr} for the \RDptr\ distributions in central \pbpb\ collisions beyond $r = 0.6$ from the jet axis suggests that the medium response to the jet is smaller than predicted for $r > 0.6$.


\begin{figure}[h]
\centerline{
\begin{tabular}{ccc}
\includegraphics[width=0.36\textwidth]{figures/results/RDpT_dR_jet7_cent0} &
\includegraphics[width=0.36\textwidth]{figures/results/RDpT_dR_jet7_cent3} &
\includegraphics[width=0.36\textwidth]{figures/results/RDpT_dR_jet7_cent5} \\
\includegraphics[width=0.36\textwidth]{figures/results/RDpT_dR_jet9_cent0} &
\includegraphics[width=0.36\textwidth]{figures/results/RDpT_dR_jet9_cent3} &
\includegraphics[width=0.36\textwidth]{figures/results/RDpT_dR_jet9_cent5} \\
\end{tabular}
}
\caption{Ratios of \Dptr\ distributions in \PbPb\ and \pp\ collisions as a function of angular distance $r$ for \ptjet\ of 126 to 158~\GeV\ (top) and of 200 to 251~\GeV\ (bottom) for seven \pt\ selections.
Different centrality selections are shown: 0--10\% (left), 30--40\% (middle), 60--80\% (right).
The vertical bars on the data points indicate statistical uncertainties while the shaded boxes indicate systematic uncertainties.
The widths of the boxes are not indicative of the bin size and the points are shifted horizontally for better visibility.}
\label{fig:rdptr}
\end{figure}


%-------------------------------------------------------------------------------
\section{Discussion}
\label{sec:discussion}
% !TEX root = trkjet.tex

The centrality dependence of \RDptr\ for two charged-particle \pt\ intervals: 1.0 -- 1.6~\GeV\ and \mbox{6.3 -- 10.0~\GeV}, and two different \ptjet\ ranges: 126 -- 158 GeV and 200 -- 251 GeV, is presented in Figure~\ref{fig:centdep}. The magnitude of these modifications decreases for more peripheral collisions and \RDptr\ approaches unity in 60 -- 80\% central collisions for both \pt\ ranges, across the entire $ r < 0.8$ range under investigation. 


\begin{figure}[h]
\centerline{
         \begin{tabular}{cc}
            \includegraphics[width=0.5\textwidth]{figures/results/RDpT_final_dR_CONF_data_cent_trk2_6_jet7.pdf} & 
            \includegraphics[width=0.5\textwidth]{figures/results/RDpT_final_dR_CONF_data_cent_trk2_6_jet9.pdf} \\
      \end{tabular}
      }
\caption{Ratios of \Dptr\ distributions for \ptjet\ of 126 to 158~\GeV\ (left) and of 200 to 251~\GeV\ (right) in \PbPb\ collisions to \pp\ collisions as a function of angular distance $r$ for two \pt\ selections and six centrality intervals (\pt\ selections are shown by closed and open symbols). The vertical bars on the data points indicate statistical uncertainties while the shaded boxes indicate systematic uncertainties. There are no uncertainties on the \rvar\ values, the finite widths of the shaded boxes are purely aesthetic.}
\label{fig:centdep}
\end{figure}


The \ptjet\ dependence of \RDptr\ for two \pt intervals: 1.0 -- 1.6~\GeV\ and \mbox{6.3 -- 10.0~\GeV}, in 0 -- 10\% central \pbpb\ collisions is presented in Figure~\ref{fig:ptjetdep}. A statistically significant trend of increasing \RDptr\ with increasing \ptjet\ is observed for $0.1 < r < 0.25$ for low pt particles. The higher-\pt\ charged particles have \RDptr\ values that decrease with increasing \rvar; no significant dependence on \ptjet\ is observed. 

\begin{figure}[ht]
\centerline{
\includegraphics[width=0.8\textwidth]{figures/results/RDpT_final_ratio_dR_CONF_data_trk2_6_cent0.pdf} 
}
\caption{\RDptr\ as a function of \rvar\ for 0--10\% collisions for charged particles with 1.6~$< \pt <$~2.5~\GeV\
(closed symbols) and 6.3~$< \pt <$10.0~\GeV\ (open symbols) for four \ptjet\ selections: 126--158~\GeV, 158--200~\GeV,
200--251~\GeV, and 251--316~\GeV. There are no uncertainties on the \rvar\ values, the finite widths of the shaded boxes are purely aesthetic.}
\label{fig:ptjetdep}
\end{figure}



Differences between the \Dptr\ distributions in \pbpb\ and \pp\ are presented as a function of $r$ for different \pt\ selections in 0 -- 10\% central collisions in Figure~\ref{fig:deltadptr}. 
These distributions indicate an excess (depletion) in the charged-particle yield for \pbpb\ collisions compared to \pp\ collisions for charged particles with low (high) \pt\. This excess ranges from 0.5 to 4 particles at 1 \GeV\ while the depletion is at most 0.5 particles at 10 \GeV. The excess is observed to increase with increasing \ptjet\ for low \pt\ particles.


These observations are in agreement with the previous measurement of jet fragmentation functions \cite{Chatrchyan:2014ava, Sirunyan:2018jqr, Aaboud:2017bzv, PhysRevC.98.024908} and may indicate the dependence of the response of the hot dense matter to the momentum of a jet passing through it. 

\begin{figure}
\centering{
\begin{tabular}{cc}
	 \includegraphics[width=0.45\textwidth]{results/DeltaDpT_final_ratio_dR_CONF_data_jet7_cent0} &
	 \includegraphics[width=0.45\textwidth]{results/DeltaDpT_final_ratio_dR_CONF_data_jet8_cent0} \\
	 \includegraphics[width=0.45\textwidth]{results/DeltaDpT_final_ratio_dR_CONF_data_jet9_cent0} &
	 \includegraphics[width=0.45\textwidth]{results/DeltaDpT_final_ratio_dR_CONF_data_jet10_cent0} \\
\end{tabular} }
   \caption{$\Delta \RDptr$ as a function of \rvar\ in central collisions for all \pt\ ranges in four \ptjet\ selections: 126--158~\GeV, 158--200~\GeV, 200--251~\GeV, and 251--316~\GeV. }
      \label{fig:deltadptr}
\end{figure}


\begin{figure}
\centering{
\begin{tabular}{cc}
	 \includegraphics[width=0.45\textwidth]{results/zRDpT_final_ratio_dR_CONF_data_jetpT_trk2_cent0} &
	 \includegraphics[width=0.45\textwidth]{results/zRDpT_final_ratio_dR_CONF_data_jetpT_trk2_cent5} \\
	 \includegraphics[width=0.45\textwidth]{results/zRDpT_final_ratio_dR_CONF_data_jetpT_trk5_cent0} &
	 \includegraphics[width=0.45\textwidth]{results/zRDpT_final_ratio_dR_CONF_data_jetpT_trk5_cent5} \\
\end{tabular} }
   \caption{\RDptr as a function of \ptjet\ for charged particles with $1.0 < \pt < 1.6$ GeV (top) and $4.0 < \pt < 6.3$ GeV (bottom) at different distances from the jet axis and in 0--10\% central (left) and 60--80\% peripheral (right) \pbpb\ collisions.}
      \label{fig:rdptr_jetpt}
\end{figure}



\begin{figure}
\centering{
\begin{tabular}{cc}
	 \includegraphics[width=0.5\textwidth]{results/xRDpT_final_ratio_dR_CONF_data_trkpT_jet7_dR0} &
	 \includegraphics[width=0.5\textwidth]{results/xRDpT_final_ratio_dR_CONF_data_trkpT_jet9_dR0} \\
	 \includegraphics[width=0.5\textwidth]{results/xRDpT_final_ratio_dR_CONF_data_trkpT_jet7_dR3} &
	 \includegraphics[width=0.5\textwidth]{results/xRDpT_final_ratio_dR_CONF_data_trkpT_jet9_dR3} \\
	 \includegraphics[width=0.5\textwidth]{results/xRDpT_final_ratio_dR_CONF_data_trkpT_jet7_dR10} &
	 \includegraphics[width=0.5\textwidth]{results/xRDpT_final_ratio_dR_CONF_data_trkpT_jet9_dR10} \\
\end{tabular} }
   \caption{\RDptr\ for central \pbpb\ collisions as a function of \pt\ for $126 < \ptjet < 158$ GeV (left) and $200 < \ptjet < 251$ GeV (left) for three different distances from the jet axis: $0.00 < \rvar < 0.05$ (top), $0.15 < \rvar < 0.20$ (middle), $0.70 < \rvar < 0.80$ (bottom)}
      \label{fig:xrdptr}
\end{figure}




\FloatBarrier

%-------------------------------------------------------------------------------
\section{Summary}
\label{sec:summary}
% !TEX root = trackjet.tex

This paper presents a measurement of the yields of charged particle distributions, \Dptr\, inside and around \RFour\ \antikt\ jets with $|\yjet| <$1.7, up to a distance of $R = 0.8$ from the jet axis. The yields are measured for \ptjet\ from 126 to 316~\GeV\ in \PbPb\ and \pp\ collisions at 5.02~\TeV\ as a function of charged particle \pt\ and the angular distance \rvar\ between the jet axis and charged particle.

Centrality dependent modifications to the yields, when compared to those measured in \pp\ collisions, are observed. The magnitude of these modifications increases with increasing collision centrality. 
The \RDptr\ distributions for charged particles with $\pt <$~4~\GeV\ 
are above unity and 
grow with increasing angular separation up to $r \sim0.3$, showing weak to no dependence on $r$ in the interval 0.3~$< \rvar <$~0.6 followed with a small decrease in the enhancement between 0.6~$< \rvar <$~0.8
For charged particles with $\pt >$~4~\GeV, a suppression in \RDptr\ is observed, and the 
distributions decrease with increasing
\rvar\ for $\rvar < $~0.3, with no \rvar\ dependence for $r>0.3$. 
These results show a hint of broadening of the \Dptr\ distribution for low \pt\ particles inside the jet
in central \pbpb\ collisions compared to those in \pp\ collisions while for higher \pt\ particles
angular distributions are narrower in \pbpb\ collisions compared to \pp\ collisions.
For all charged-particle \pt\ values the \RDptr\ values are greater than or equal to unity for
small \rvar\ values (inside the core of the jet).
Between $0.1 < r < 0.25$, a statistically significant trend of increasing \RDptr\ with increasing \ptjet\ is observed for low-\pt\ particles. No significant \ptjet\ dependence is seen for particles  with $\pt >$~4~\GeV.

These measurements provide insight into the differential distributions of charged particles within jets as compared to the inclusive measurement of jet fragmentation functions. % and are important
%for understanding how jets interact with the QGP.
They provide new information about our understanding of the physics of soft
gluon radiation and the response of the QGP to jets.

%-------------------------------------------------------------------------------
\section*{Acknowledgements}
\input{acknowledgements/Acknowledgements}
%-------------------------------------------------------------------------------

% All figures and tables should appear before the summary and conclusion.
% The package placeins provides the macro \FloatBarrier to achieve this.
% \FloatBarrier


%The \texttt{atlaslatex} package contains the acknowledgements that were valid 
%at the time of the release you are using.
%These can be found in the \texttt{acknowledgements} subdirectory.
%When your ATLAS paper or PUB/CONF note is ready to be published,
%download the latest set of acknowledgements from:\\
%\url{https://twiki.cern.ch/twiki/bin/view/AtlasProtected/PubComAcknowledgements}


%-------------------------------------------------------------------------------
%\clearpage
%\appendix
%\part*{Appendix}
%\addcontentsline{toc}{part}{Appendix}
%-------------------------------------------------------------------------------


%-------------------------------------------------------------------------------
% If you use biblatex and either biber or bibtex to process the bibliography
% just say \printbibliography here
\printbibliography
% If you want to use the traditional BibTeX you need to use the syntax below.
% \bibliographystyle{obsolete/bst/atlasBibStyleWoTitle}
% \bibliography{trkjet,bib/ATLAS,bib/CMS,bib/ConfNotes,bib/PubNotes}
%-------------------------------------------------------------------------------

%-------------------------------------------------------------------------------
% Author list - comment in this line when you are ready to include it
% \clearpage
% \input{atlas_authlist}
%-------------------------------------------------------------------------------

%-------------------------------------------------------------------------------
% Auxiliary material - comment out the following line if you do not have any
% !TEX root = trkjet.tex

\part*{Auxiliary material}
\addcontentsline{toc}{part}{Auxiliary material}
%-------------------------------------------------------------------------------

%In an ATLAS paper, auxiliary plots and tables that are supposed to be made public 
%should be collected in an appendix that has the title \enquote{Auxiliary material}.
%This information will appear on the public webpage, but will not be included
%in the document submitted to arXiv and to the journal.


\begin{figure}[h]
\includegraphics[width=1.0\textwidth]{figures/results/DpT_dR_jet7.pdf}
\caption{ \Dptr\ distributions as a function of \rvar\ for different \pt\ ranges in 126--158 GeV jets. The open markers are for \pp\ collisions and the solid markers are for \pbpb\ collisions. The different panels refer to different centrality selections}
\label{fig:fullset_dptr_j7}
\end{figure}

\begin{figure}[h]
\includegraphics[width=1.0\textwidth]{figures/results/DpT_dR_jet8.pdf}
\caption{ \Dptr\ distributions as a function of \rvar\ for different \pt\ ranges in 158--200 GeV jets. The open markers are for \pp\ collisions and the solid markers are for \pbpb\ collisions. The different panels refer to different centrality selections}
\label{fig:fullset_dptr_j8}
\end{figure}

\begin{figure}[h]
\includegraphics[width=1.0\textwidth]{figures/results/DpT_dR_jet9.pdf}
\caption{ \Dptr\ distributions as a function of \rvar\ for different \pt\ ranges in 200--251 GeV jets. The open markers are for \pp\ collisions and the solid markers are for \pbpb\ collisions. The different panels refer to different centrality selections}
\label{fig:fullset_dptr_j9}
\end{figure}

\begin{figure}[h]
\includegraphics[width=1.0\textwidth]{figures/results/DpT_dR_jet10.pdf}
\caption{ \Dptr\ distributions as a function of \rvar\ for different \pt\ ranges in 251--316 GeV jets. The open markers are for \pp\ collisions and the solid markers are for \pbpb\ collisions. The different panels refer to different centrality selections}
\label{fig:fullset_dptr_j10}
\end{figure}


\begin{figure}[h]
\includegraphics[width=1.0\textwidth]{figures/results/RDpT_dR_jet7.pdf}
\caption{The \RDptr\ distributions as a function of \rvar\ for different \pt\ selections in 126--158 GeV jets. The different panels refer to different centrality selections.}
\label{fig:fullset_rptr_j7}
\end{figure}

\begin{figure}[h]
\includegraphics[width=1.0\textwidth]{figures/results/RDpT_dR_jet8.pdf}
\caption{The \RDptr\ distributions as a function of \rvar\ for different \pt\ selections in 158--200 GeV jets. The different panels refer to different centrality selections.}
\label{fig:fullset_rptr_j8}
\end{figure}

\begin{figure}[h]
\includegraphics[width=1.0\textwidth]{figures/results/RDpT_dR_jet9.pdf}
\caption{The \RDptr\ distributions as a function of \rvar\ for different \pt\ selections in 200--251 GeV jets. The different panels refer to different centrality selections.}
\label{fig:fullset_rptr_j9}
\end{figure}

\begin{figure}[h]
\includegraphics[width=1.0\textwidth]{figures/results/RDpT_dR_jet10.pdf}
\caption{The \RDptr\ distributions as a function of \rvar\ for different \pt\ selections in 251--316 GeV jets. The different panels refer to different centrality selections.}
\label{fig:fullset_drtr_j10}
\end{figure}

%-------------------------------------------------------------------------------

%-------------------------------------------------------------------------------
% Extra tables etc. for HepData - comment in the following line if you have any
% \include{trkjet-hepdata}
%-------------------------------------------------------------------------------

\end{document}
