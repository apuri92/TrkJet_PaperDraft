% !TEX root = trkjet.tex

The analysis procedure is similar to that in Ref.~\cite{Aaboud:2018hpb} with the additional requirement of being done differentially in \rvar. Measured tracks are associated with a reconstructed jet if they fall within $\Delta R < 0.8$ of the jet axis and are constructed as:

\begin{align*}
\dfrac{\fd^{2} \nchmeas }{ \fd \pttrk \fd r} = \frac{1}{\varepsilon(\pttrk,\etatrk)} \frac{\Delta \Nch( \pttrk, r)}{ \Delta\pttrk \Delta r}
\end{align*}

where $\Delta \Nch (\pttrk, r)$ represents the number of tracks within a given \pttrk\ and $r$ range. The efficiency correction is applied as a $1/\varepsilon(\pttrk,\etatrk)$ weight on a track-by-track basis, assuming $\pttrk = \pTtrue$. While that assumption is not strictly valid, the efficiency varies sufficiently slowly with $\pTtrue$ that the error
introduced by this assumption is less than 1\%. It is further corrected for by the Bayesian unfolding procedure described later in this section.

The measured track yields need to be corrected for the UE, fake tracks and secondaries. In \pp\ collisions, the UE contribution from hard scatterings not associated with jet production is negligible. The contributions from fake tracks and secondary charged particles are estimated from MC samples and subtracted. This procedure is similar to that applied in previous measurements~\cite{Aaboud:2017tke,Aaboud:2018hpb}.

For \pbpb\ collisions, the UE, fake track, and secondary contributions are estimated together in a two step process: first, MC overlay is used to generate \etajet--\phijet\ maps of the average number of charged particles in a given annulus around a reconstructed jet. This is done for charged particles without a truth match and as a function of \ptjet, \etajet, \phijet, angle of the jet to the second order event plane\footnote{The second order event plane angle $\Psi_2$ is determined on an event-by-event basis by a standard method using the $\phi$ variation of transverse energy in the FCal \cite{ATLAS:2012at}} $ \mathrm{d}\Psi_{\mathrm{jet}}$, \rvar, \pttrk, and centrality. In the second step, the \etajet--\phijet\ maps are used to generate the UE distribution for jets with a given \etajet, \phijet, and $\mathrm{d}\Psi_{\mathrm{jet}}$. This distribution includes fakes, and is given by \mbox{$\fd^2 \nch^{\mathrm{UE+Fake}}(\pttrk, r) / \fd \pttrk \fd r$}. The yields decrease with decreasing collision centrality, increasing \pttrk, and increasing azimuthal distance from $\Psi_2$. The subtracted distributions are then given by 

\begin{align*}
\frac{\fd^2 \nchsub }{ \fd \pttrk \fd r } &=  \frac{\fd^2 \nchmeas }{ \fd \pttrk \fd r} -  \frac{ \fd^2 \nch^{\mathrm{UE+Fake}}(r)  }{ \fd \pttrk \fd r} 
\end{align*}

Figure~\ref{fig:UEsize} shows the ratio of the charged-particle distributions before and after the subtraction of the UE, fake tracks, and secondaries,
%the charged-particle distributions prior to the UE and fake track subtraction, $ \fd^2 \nchmeas / \fd \pttrk \fd r$, divided by the distributions after the subtraction, $ \fd^2 \nchsub / \fd \pttrk \fd r $
 as a function of \rvar\ for different \pttrk\ intervals and $126 < \ptjet < 158$ \GeV\ for six centrality selections. The largest UE contribution is for 1.0~\GeV\ charged particles at large values of \rvar\ in central collisions, with the background being approximately 100 times the signal, and slowly decreasing with increasing \ptjet. It rapidly decreases for more peripheral collisions, larger \pttrk\ and smaller \rvar. In addition, due to the steeply falling nature of the jet \pt\ spectra, the smearing due to jet energy resolution leads to a net migration of jets from lower \ptjet\ to higher \ptjet\ values, such that a jet reconstructed with a given transverse momentum will correspond, on average, to a lower truth jet \pT.  This ``up-feeding'' induces a difference between the UE yields determined using the MC overlay events and the actual UE contribution to reconstructed jets. The magnitude of this difference is centrality dependent, and can be seen clearly for charged particles with $\pt > 10$ GeV.

\begin{figure}
\centerline{
 \includegraphics[width=0.95\textwidth]{figures/performance/UE_B2S_single_0.pdf} }
\caption{ Ratio of the raw charged particle distributions to those after the subtraction of the UE and fake tracks as a function of \rvar\ for different \pttrk\ intervals, six centrality selections and for \ptjet\ between 126--158~\GeV.}
\label{fig:UEsize}
\end{figure}


%The fraction of fake tracks is found to be below 2\% of the tracks that pass the selection in all track and jet kinematic regions in this analysis.

To remove the effects of the bin migration due to the jet energy and track momentum resolution, the subtracted $\fd^2 \nchsub /\fd\pttrk \fd r$ distributions are corrected by a two-dimensional Bayesian unfolding~\cite{DAgostini:1994zf}
in \pttrk\ and \ptjet\ as implemented in the RooUnfold package~\cite{Adye:2011gm}.  
Two-dimensional unfolding is used because the calorimetric jet energy response depends on the fragmentation pattern of the jet~\cite{Aad:2011he}.
Four-dimensional response matrices are created from the \pp\ and \pbpb\ MC samples using the generator-level and reconstructed \ptjet, and the generator-level and reconstructed charged-particle \pttrk. They are corrected for tracking efficiencies and are evaluated in bins of \rvar\ and centrality. The Bayesian procedure requires a choice in the number of iterations.
Additional iterations reduce the sensitivity to the choice of prior, but may
amplify statistical fluctuations in the distributions.
After four iterations the 
charged particle distributions are found to be stable for both the \PbPb\ and \pp\ data.
A separate one-dimensional Bayesian unfolding is used to correct the measured \ptjet\ spectra that are used to normalize the unfolded charged particle distributions.
To achieve better correspondence with the data, the response matrices for both the one and two dimensional unfolding are reweighed so that the charged particle and jet distributions match the shapes in the reconstructed data.

An independent bin-by-bin unfolding procedure is also used to correct for migrations originating from the finite jet and track angular resolutions. Two corresponding \Dptr\ distributions are evaluated in MC samples, one using truth jets\footnote{Truth jets are reconstructed by applying the \antikt\ algorithm to stable final-state particles from MC generators like \PYTHIA. Particles are required to have a lifetime of $c\tau > 10$ mm and muons, neutrinos, and particles from pile-up activity are excluded} and primary particles and the other using reconstructed jets and charged particles with their reconstructed \pt\ replaced by generator-level transverse momentum, \pTtrue. The ratio of these two MC distributions provides a correction factor which is then applied to the data. 

The final particle-level corrected distributions, normalized by the area of the annulus under question are defined as:

\begin{align*}
   \Dptr = \frac{1}{N_\mathrm{jet}^\mathrm{unfolded}} \frac{1}{A(r)} \frac{\fd^2 \nchunf(r)}{\fd \pt \fd r},
 \end{align*}
where $N_\mathrm{jet}^\mathrm{unfolded}$ is the unfolded number of jets in a given \ptjet\ interval, and \nchunf\  is the unfolded yield of charged particles with a given \pt in an annulus of area $A$ at a distance \rvar.

The performance of the full analysis procedure is validated in the MC samples by comparing the fully corrected charged particle distributions to the generator-level distributions. Good closure (< 4 \%) is seen for charged particles with $\pt < 10$ GeV in both the \pp\ and \pbpb\ collision systems. The non-closure is taken as an additional systematic uncertainty as discussed in Section~\ref{sec:systematics}.
It is to be noted that adding or removing particles carrying a large fraction of the jet momentum near the edge of the jet can significantly alter its
reconstructed momentum and position; this instability leads to some non-closure in the analysis procedure for particles with $\pt > 10$ GeV in jets with $\ptjet < 200$ GeV. 
Results are presented where the non-closure in the \pp\ MC sample is less than 5\%.

%, excluding the following regions of phase space: 6--10 GeV tracks above $\rvar > 0.3$, 10--25 GeV tracks above $\rvar > 0.3$, and 25--63 GeV tracks above $\rvar > 0.2$ for 126--158 GeV jets; 10--25 GeV tracks above $\rvar > 0.4$, and 25--63 GeV tracks above $\rvar > 0.3$ for 158--200 GeV jets; 25--63 GeV tracks above $\rvar > 0.3$ for 200--251 GeV jets.



